\documentclass[report.tex]{subfile}

\begin{document}

\section{Result}
A classifier has been implemented and trained as described in the previous
section. The total training set was 1000 digits, the number of digits was 700,
100 and 200 in the training, validation and testing set respectively.

\subsection{Validation}
The training phase produced 16 different classifiers, the classifier that
performed the best on the validation set was the one with 13 neurons in the
hidden layer. It correctly classified 95\% of the samples in the validation
set.

\subsection{Testing}
The selected classifier from the validation phase was tested on the training
set. The resulting confusion matrix is shown in \ref{fig:confusion}. The
classifier classified 96.5\% of the digits in the testing set correctly.
Previous runs created classifiers with similar results, ranging from 92\% to
97\%.

7 of the 200 digits in the testing set were misclassified. These are shown in
figure \ref{fig:incorrect}.

\begin{figure}
    \hspace*{-2cm}
    \resizebox{1.2\textwidth}{!}{\input{build/fig/incorrect.tex}}
    \caption{The guess from the classifier vs the label for all incorrectly
    classified digits from the test set (N=200).}
    \label{fig:incorrect}
\end{figure}

\begin{table}
    \centering
    \begin{tabular}{rr|cccccccccc}
        & & \multicolumn{10}{c}{Actual digit}\\
        \multirow{10}{*}{\rotatebox{90}{Guessed digit}}
        &   & 0 & 1 & 2 & 3 & 4 & 5 & 6 & 7 & 8 & 9\\\hline
        & 0 & 19 & 0 & 0 & 0 & 0 & 1 & 0 & 0 & 0 & 0\\
        & 1 & 0 & 20 & 0 & 0 & 0 & 0 & 0 & 0 & 0 & 0\\
        & 2 & 0 & 0 & 20 & 0 & 0 & 0 & 0 & 0 & 0 & 0\\
        & 3 & 0 & 0 & 0 & 20 & 0 & 0 & 0 & 0 & 0 & 0\\
        & 4 & 0 & 0 & 0 & 0 & 20 & 0 & 1 & 0 & 0 & 0\\
        & 5 & 1 & 0 & 0 & 0 & 0 & 18 & 0 & 1 & 0 & 0\\
        & 6 & 0 & 0 & 0 & 0 & 0 & 1 & 19 & 0 & 0 & 0\\
        & 7 & 0 & 0 & 0 & 0 & 0 & 0 & 0 & 17 & 0 & 0\\
        & 8 & 0 & 0 & 0 & 0 & 0 & 0 & 0 & 0 & 20 & 0\\
        & 9 & 0 & 0 & 0 & 0 & 0 & 0 & 0 & 2 & 0 & 20\\
    \end{tabular}
    \caption{Confusion matrix from classifications of the test dataset.}
    \label{fig:confusion}
\end{table}

\subsection{Differing setups}
Further runs were made with different setups, changing different parameters but
keeping the method described. The parameters and results from the validation
and testing are shown in \ref{fig:runs}.

\begin{table}[h]
    \centering
    \begin{tabular}{rrrrr|rrr}
        dilation & set distribution & W & $\rho$ & iterations &
        hidden & F1 (validation) & F1 (test) \\\hline

        yes & (70, 10, 20) & 20 & 0.0001  & 1000 & 13 & 95.0\% & 96.5\% \\
        yes & (70, 10, 20) & 20 & 0.0001  & 1000 & 24 & 98.0\% & 92.5\% \\
        yes & (70, 10, 20) & 20 & 0.0001  & 1000 & 21 & 98.0\% & 94.5\% \\
        no  & (70, 10, 20) & 20 & 0.0001  & 1000 & 13 & 84.0\% & 89.5\% \\
        no  & (70, 10, 20) & 20 & 0.0001  & 1000 & 11 & 93.0\% & 90.5\% \\
        no  & (70, 10, 20) & 20 & 0.0001  & 1000 & 14 & 93.0\% & 91.0\% \\
        yes & (70, 10, 20) & 17 & 0.0001  & 1000 & 21 & 99.0\% & 92.5\% \\
        yes & (70, 10, 20) & 25 & 0.0001  & 1000 & 21 & 97.0\% & 91.5\% \\
        yes & (70, 10, 20) & 35 & 0.0001  & 1000 & 13 & 96.0\% & 91.5\% \\
        yes & (70, 20, 10) & 20 & 0.0001  & 1000 & 18 & 92.0\% & 96.0\% \\
        yes & (60, 30, 10) & 20 & 0.0001  & 1000 & 14 & 99.3\% & 94.0\% \\
        yes & (50, 40, 10) & 20 & 0.0001  & 1000 & 21 & 95.5\% & 94.0\% \\
        yes & (70, 10, 20) & 20 & 0.0003  & 1000 & 12 & 97.0\% & 94.0\% \\
        yes & (70, 10, 20) & 20 & 0.00005 & 1000 & 12 & 90.0\% & 90.5\% \\
        yes & (70, 10, 20) & 20 & 0.00005 & 3000 & 20 & 97.0\% & 95.5\% \\
        yes & (70, 10, 20) & 20 & 0.00001 & 1000 & 12 & 99.0\% & 94.0\% \\
    \end{tabular}
    \caption{Results of various runs with different setups.}
    \label{fig:runs}
\end{table}

\end{document}
