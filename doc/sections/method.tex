\documentclass[report.tex]{subfile}

\begin{document}

\section{Method}

\subsection{Data subsets partitioning}

\subsection{Preprocessing}
The input data for each data is given as a list of 3D-coordinates that together
form a curve in the shape of a digit. As the digits are aligned along the plane
of the XY axes, the Z-coordinate is simply ignored and the curves are treated
as 2D only. The digits are first translated and scaled to a common location and
size and then rasterized into a grayscale image. The grayscale image is then
dilated so that the digits become thicker. The intensity of each pixel is
finally treated as a large vector. Each pixel represents an input neuron in the
neural network.

\subsubsection{Scaling}
\begin{figure}
    \centering
    \begin{subfigure}[c]{0.3\textwidth}
        \resizebox{\textwidth}{!}{\input{build/fig/scale_before1.tex}}
        \caption{Six \#1, raw}
    \end{subfigure}%
    {\Huge $\rightarrow$}
    \begin{subfigure}[c]{0.3\textwidth}
        \resizebox{\textwidth}{!}{\input{build/fig/scale_after1.tex}}
        \caption{Six \#1, scaled}
    \end{subfigure}\\
    \begin{subfigure}[c]{0.3\textwidth}
        \resizebox{\textwidth}{!}{\input{build/fig/scale_before2.tex}}
        \caption{Six \#2, raw}
    \end{subfigure}%
    {\Huge $\rightarrow$}
    \begin{subfigure}[c]{0.3\textwidth}
        \resizebox{\textwidth}{!}{\input{build/fig/scale_after2.tex}}
        \caption{Six \#2, scaled}
    \end{subfigure}
    \caption{Two sixes before and after rescaling.}
    \label{fig:scaling}
\end{figure}
Because the digits are written at different locations and in different sizes,
rescaling and translation is performed to make similar shaped images
comparable. Only the shape of the digit is interesting and the location and
scale can therefore be changed without losing any useful information. Figure
\ref{fig:scaling} shows two sixes being scaled.

\subsubsection{Rasterization}
\begin{figure}
    \centering
    \begin{subfigure}[c]{0.3\textwidth}
        \vspace*{-3.5mm}
        \resizebox{\textwidth}{!}{\input{build/fig/rasterize_0.tex}}
        \vspace*{-10mm}
        \caption{Scaled}
    \end{subfigure}%
    {\Large $\rightarrow$}
    \begin{subfigure}[c]{0.3\textwidth}
        \centering
        \resizebox{0.8\textwidth}{!}{%
            \includegraphics{build/fig/rasterize_1.png}%
        }
        \vfill
        \caption{Rasterized}
    \end{subfigure}%
    {\Large $\rightarrow$}
    \begin{subfigure}[c]{0.3\textwidth}
        \centering
        \resizebox{0.8\textwidth}{!}{%
            \includegraphics{build/fig/rasterize_2.png}%
        }
        \vfill
        \caption{Dilated}
    \end{subfigure}%
    \caption{Rasterization of a two with some noise.}
    \label{fig:raster_noise}
\end{figure}
Because the digits are written at different locations and in different sizes,
rescaling and translation is performed to make similar shaped images
comparable. Only the shape of the digit is interesting and the location and
scale can therefore be changed without losing any useful information. Figure
\ref{fig:scaling} shows two sixes being scaled.
When the digits are located in the same position and are of similar scale the
digits can be rasterized. The rasterization is performed by rasterizing each
line of the curve. The rasterization is performed by using Xiaolin Wu's line
drawing algorithm. The algorithm is described in Wu's article in Computer
Graphics. \cite{wu-line}

Each rasterized curve is then added to the final image by taking the maximum
intensity of the previous pixel and the pixel of the curve.

Finally, dilation is performed to make the digits thicker. The dilation is
performed by applying a two-dimensional convolution to the image with the
matrix
\begin{equation*}
    D =
    \begin{bmatrix}
        0.5 & 0.8 & 0.5 \\
        0.8 &   1 & 0.8 \\
        0.5 & 0.8 & 0.5
    \end{bmatrix}.
\end{equation*}
In this way, the inner pixels of digits will be 1 and the edges will be between
0 and 1, which could be useful for the classifier because the same digit
usually have edges in around the same location.

One good thing about rasterization is that it decreases the impact of noise as
is demonstrated in figure \ref{fig:raster_noise}. Also, duplicate overlapping
lines will look the same as if there was only one line.

\subsection{Training of neural network}
The classifier was implemented using a multi-layer perceptron with one hidden
layer. The input layer consists of all the pixels of the image resulting from
the preprocessing of the input.

The single hidden layer contains 10 neurons.  A lower number appears to
decrease the performance while increasing only increases the training time.

\subsection{Evaluation}

\end{document}
